\documentclass{beamer}
\usepackage[czech]{babel}
\usepackage[utf8]{inputenc}
\usepackage{pmboxdraw}
\usetheme {darmstadt}
\date{}
\title{Překladač programovacího jazyka Zyba}
\author{Richard Blažek}
\begin{document}
\begin{frame}
\titlepage
\end{frame}
\begin{frame}{Cíle práce}
\begin{itemize}
    \item Vyřešit nepříjemnosti s PHP
    \item Navrhnout vlastní programovací jazyk Zyba
    \item Napsat překladač ze Zyby do PHP
\end{itemize}
\end{frame}
\begin{frame}{Co Zyba umí}
\begin{itemize}
    \item Statické typování s typovou inferencí
    \item Imperativní i funkcionální programování
    \item Pole, slovníky, záznamy, lambda funkce
    \item Moduly, možnost zvolit co exportovat, jmenné prostory
    \item Importy z PHP i z webu
\end{itemize}
\end{frame}
\begin{frame}{Jak se překládá}
\begin{itemize}
    \item Lexikální a syntaktická analýza
    \item Sémantická analýza a typová kontrola
    \item Vyřešení importů a závislostí mezi moduly
    \item Generování PHP kódu
\end{itemize}
\end{frame}
\begin{frame}{Co si z toho odnést}
\begin{itemize}
    \item Zyba umí dost věcí na to, abyste mohli vidět online prezentaci, kdyby fungoval internet
    \item Další vývoj -- interoperabilita s PHP, generické programování
    \item Je lepší mít vždycky záložní plán
\end{itemize}
\end{frame}
\end{document}
\documentclass[a4paper,12pt]{article}
\usepackage[margin=25mm]{geometry}
\usepackage[czech]{babel}
\usepackage[utf8]{inputenc}
\usepackage{amsmath}
\usepackage{amssymb}
\usepackage{graphicx}
\usepackage{fancyhdr}
\usepackage[backend=bibtex,sorting=none]{biblatex}
\usepackage{tikz}
\usepackage{xcolor}
\usepackage{float}
\usepackage{hyperref}
\usepackage[nounderscore]{syntax}
\usepackage[T1]{fontenc}
\usepackage{textcomp}
\usepackage{array}
\usepackage{multirow}

\definecolor{keyword}{RGB}{0, 64, 255}

\addbibresource{SOC.bib}

\usetikzlibrary{shapes.geometric, arrows}
\tikzstyle{block} = [rectangle, minimum width=3cm, minimum height=1cm,text centered, draw=black, fill=blue!25]
\tikzstyle{arrow} = [thick,<-,>=stealth]
\def\circumflex{\char`\^}
\def\quote #1{"#1"}
\def\keyword #1{\color{keyword}#1\color{black}}
\def\emphasis #1{\textbf{\texttt{\quote{#1}}}}
\renewcommand{\labelitemii}{$\circ$}
\renewcommand{\labelitemiii}{$-$}
\makeatletter
\newcount\my@repeat@count
\newcommand{\repeatchar}[2]{
  \begingroup
  \my@repeat@count=\z@
  \@whilenum\my@repeat@count<#1\do{#2\advance\my@repeat@count\@ne}
  \endgroup
}
\makeatother

\author{Richard Blažek}
\setlength{\headheight}{15pt}
\pagestyle{fancy}
\fancyhead{}
\fancyhead[R]{Překladač jazyka Zyba}
\fancyhead[L]{Richard Blažek}
\fancyfoot{}
\fancyfoot[R]{\thepage}
\fancyfoot[L]{Kapitola \thesection}
\setlength\parindent{0em}
\setlength\parskip{0.8em}
\def\arraystretch{1.2}

\begin{document}
\begin{titlepage}
    \begin{center}

	\vspace*{3.5cm}            
	\Huge
	\textbf{Překladač jazyka Zyba}
            
	\vspace{0.5cm}
	\LARGE
	Staticky typovaný jazyk kompilovaný do~PHP
        
	\vspace*{1.5cm}
	\Huge
	\textbf{Zyba language compiler}
            
	\vspace{0.5cm}
	\LARGE
	Language with static typing transpiled into PHP
            
	\vfill
            
	\large
        Středoškolská odborná činnost, rok 2022\\
	Richard Blažek\\
	Gymnázium Brno, třída Kapitána Jaroše 14
    \end{center}
\end{titlepage}
\thispagestyle{empty}
\Large\textbf{Prohlášení}\normalsize

Prohlašuji, že jsem svou závěrečnou maturitní práci vypracoval samostatně a použil jsem pouze prameny a literaturu uvedené v~seznamu bibliografických záznamů.

Prohlašuji, že tištěná verze a elektronická verze závěrečné maturitní práce jsou shodné.

Nemám závažný důvod proti zpřístupňování této práce v~souladu se zákonem č. 121/2000 Sb., o~právu autorském, o~právech souvisejících s~právem autorským a o~změně některých zákonů (autorský zákon) ve~znění pozdějších předpisů. 

V~Brně dne \today{} \repeatchar{40}{.}
\newpage
\thispagestyle{empty}
\Large\textbf{Poděkování}\normalsize

Tímto bych chtěl poděkovat Matěji Žáčkovi za~odborné vedení práce.
\newpage
\thispagestyle{empty}
\Large\textbf{Anotace}\normalsize

Práce se zabývá navržením jazyka Zyba a implementací překladače tohoto jazyka do~PHP, což by mělo umožnit používání tohoto jazyka na~všech webhostinzích, které podporují PHP skripty. Rovněž bude možné vyvíjet část projektu v~Zybě a část v~PHP a používat funkce z~jednoho jazyka ve~druhém.

\Large\textbf{Klíčová slova}\normalsize

programovací jazyk; překladač; transpiling; webová aplikace; PHP; Zyba

\Large\textbf{Annotation}\normalsize

The thesis is concerned with the design of the Zyba language and implementing its compiler. The compiler generates PHP code, making it possible for the language to be used on all webhostings where PHP is supported. Also, it will be possible to combine Zyba and PHP when developing web applications and use functions from one of the languages in the other one.

\Large\textbf{Keywords}\normalsize

programming lanugage; compiler; transpiling; web application; PHP; Zyba
\newpage
\thispagestyle{empty}
\tableofcontents
\newpage
\section{Úvod}
Cílem této práce je navrhnout programovací jazyk Zyba, který by měl umožnit psaní přehlednějšího kódu než PHP, a implementovat překladač ze Zyby do~PHP, aby bylo možné programy v~Zybě používat na~všech na~všech serverech s~podporou PHP skriptů. Navrhl jsem Zybu jako staticky typovaný jazyk, protože překlad z~jednoho dynamicky typovaného jazyka do~druhého by umožnil provádět při~překladu pouze syntatickou kontrolu. Vyhodnocení typů by mohlo proběhnout až za~běhu programu a Zyba by tak představovala jen alternativní syntaxi pro~PHP. Zybu jsem navrhl jako jazyk jednoduchý na~naučení, ale s~dostatečnou funkcionalitou pro~psaní webových stránek.

Překladač jsem se rozhodl napsat v~jazyce Haskell, protože umožňuje psát velmi stručné a přehledné programy. K~čitelnosti programu přispívá jednak syntaxe jazyka, v~němž se struktura programu vyjadřuje formátováním zdrojového kódu a ne oddělovači, jednak jeho striktní dodržování funkcionálního paradigmatu, které vyžaduje, aby funkce byly referenčně transparentní (tzn. bez vedlejších efektů). Navíc jeho typový systém obsahuje algebraické datové typy a umožňuje zápis rekurzivních typů, což se u překladače hodí například na~zápis syntaktického stromu.

\section{Návrh jazyka}
\subsection{Syntaxe}
Syntaxe Zyby je stejně jako u řady dalších jazyků (např. C++, Java, C\#, JavaScript) odvozená z~jazyka C, aby byla blízká ostatním programátorům, ale liší se v~řadě detailů. Všechny operátory jsou binární, zleva asociativní a mají stejnou prioritu. Práce s~poli a slovníky (tj. jejich vytváření a přístup k~prvkům a jejich změna) není záležitostí speciální syntaxe, ale provádí se zabudovanými metodami až na~úrovni sémantiky. Volání funkce je vyjádřeno hranatými závorkami. Díky těmto změnám je syntaxe jednoznačná, aniž by musel uživatel používat středníky. Středníky a čárky jsou považovány za~bílé znaky, takže je možné je používat, kde to programátor uzná za~vhodné pro~přehlednost kódu. Komentáře jsou jednořádkové a začínají křížkem (\texttt{\#}).

Následuje popis prvků syntaxe a její formální popis v~rozšířené Backus-Naurově formě.

\subsubsection{Deklarace}
Soubor v~Zybě se skládá z~deklarací. Každá deklarace buď přiřadí určité globální konstantě hodnotu určitého výrazu, nebo importuje deklarace z~jiného souboru. Importují se přitom pouze ty deklarace, které byly uvozeny slovem \texttt{export}
\begin{grammar}
<file> ::= ((\emphasis{export})? <declaration>)*

<declaration> ::= <assignment> | <import>

<assignment> ::= <name> \emphasis{=} <expression>
\end{grammar}

Importovat lze jak ze Zyby tak i z~PHP. Import ze Zyby vyžaduje cestu k~souboru, který má být importován, a název jmenného prostoru, do~kterého budou importované hodnoty spadat. Import z~PHP to vyžaduje rovněž, ale před názvem jmenného prostoru musí být napsáno slovo \texttt{php} a za~cestou k~souboru musí následovat záznam se jmény a typy importovaných hodnot. Záznam tvoří složené závorky a v~nich několik dvojic názvů (tj. jména importovaných hodnot) a výrazů (tj. jejich typy).
\begin{grammar}
<import> ::= <import-zyba> | <import-php>

<import-zyba> ::= \emphasis{import} <name> <literal-text>

<import-php> ::= \emphasis{import php} <name> <literal-text> <record>

<record> ::= \emphasis{\{} (<name> <expression>)* \emphasis{\}}
\end{grammar}

\subsubsection{Výrazy}
Výraz tvoří jeden či více podvýrazů oddělených binárními operátory. Podvýraz tvoří jednotka, kterou můžou následovat volání funkce a přístupy k~prvkům záznamu nebo jmenného prostoru. Volání funkce tvoří několik výrazů v~hranatých závorkách; přístup k~prvku tvoří tečka následovaná jménem. Zabudované metody se volají způsobem, který kombinuje syntaxi přístupu a syntaxi volání funkce: \texttt{argument1.metoda[argument2 \ldots{} argument\textsubscript{N}}, přičemž mají-li jen jeden argument, je možné závorky vynechat.
\begin{grammar}
<expression> ::= <call> (<operator> <call>)*

<subexpression> ::= <unit> (<call> | <access>)*

<call> ::= \emphasis{[} <expression>* \emphasis{]}

<access> ::= \emphasis{.} <name>
\end{grammar}

Jednotek je několik druhů: výraz v~závorkách; literál celého čísla, reálného čísla, logické hodnoty či text; záznam nebo lambda funkce.
\begin{grammar}
<unit> ::= \emphasis{(} <expression> \emphasis{)}

<unit> ::= <literal-int> | <literal-real> | <literal-bool> | <literal-text>

<unit> ::= <record>

<unit> ::= <lambda>
\end{grammar}

\subsubsection{Funkce a příkazy}
Lambda funkce začíná slovem \texttt{fun}. Následují argumenty funkce v~hranatých závorkách, výraz specifikující typ navrácené hodnoty a blok, který tvoří několik příkazů ve~složených závorkách. Argumenty funkce, pokud nějaké jsou, se zapisují po skupinách, z~nichž každá se skládá z~názvů jednoho či více argumentů následovaných dvojtečkou a výrazem specifikujícím typ těchto argumentů.
\begin{grammar}
<lambda> ::= \emphasis{fun} \emphasis{[} <arguments> \emphasis{]} <expression> <block>

<arguments> ::= (<name>+ \emphasis{:} <expression>)*

<block> ::= \emphasis{\{} <statement>* \emphasis{\}}
\end{grammar}

Příkazem může být výraz, jenž má být vyhodnocen, přiřazení, příkazy \texttt{return} a \texttt{if} nebo cykly \texttt{while} či \texttt{for}. Příkaz \texttt{return} je slovo \texttt{return} následované výrazem, jenž má být navrácená hodnota. Příkaz \texttt{if} začíná slovem \texttt{if} následovaným výrazem (podmínkou) a blokem příkazů; za~blokem můžou následovat další části, jež začínají slovy \texttt{else if} a rovněž pokračují podmínkou a blokem příkazů; poté může následovat slovo \texttt{else} následované blokem, který se provede, pokud žádná z~předchozích podmínek nebyla splněna. Cyklus \texttt{while} začíná slovem \texttt{while} následovaným výrazem (podmínkou) a blokem příkazů.
\begin{grammar}
<statement> ::= <expression> | <assignment> | <if> | <while> | <for>

<return> ::= \emphasis{return} <expression>

<if> ::= \emphasis{if} <condition> <block> (\emphasis{else if} <condition> <block>)* (\emphasis{else} <block>)?

<while> ::= \emphasis{while} <condition> <block>
\end{grammar}

Cyklus \texttt{for} má dvě formy. První z~nich se používá pro~daný počet opakování, jež je specifikován nějakým výrazem, v~tom případě po slově \texttt{for} následuje jméno konstanty, která udává, kolik opakování už proběhlo, operátor \texttt{\textless}, onen výraz a blok příkazů. Druhou lze použít pro~iteraci přes pole nebo slovník, ta se zapisuje slovem \texttt{for}, po němž následují jména dvou konstant, z~nichž první bude obsahovat index současného prvku a druhá jeho hodnotu, oddělovač\texttt{:}, výraz, jehož hodnotou je ono pole či slovník, a blok příkazů. První jméno je nepovinné, v~tom případě se hodnota indexu nebude do~žádné konstanty ukládat.
\begin{grammar}
<for> ::= \emphasis{for} <name> \emphasis{\textless} <expression> <block>

<for> ::= \emphasis{for} <name>? <name> \emphasis{:} <expression> <block>
\end{grammar}

\subsubsection{Tokeny}
Nejmenší jednotkou syntaxe jsou tokeny. Token může být jedno z~následujících:
\begin{itemize}
\item \textbf{Jméno} je posloupnost znaků, jež obsahuje pouze číslice, malá a velká písmena latinky a podtržítka a jejíž první znak není číslice. Zyba nemá klíčová slova, ale některá jména (např. \texttt{if}, \texttt{fun}, \texttt{export}) mají zvláštní význam na~určitých místech kódu, takže není vhodné je používat pro~pojmenování vlastních proměnných.
\item \textbf{Literál celého čísla} je buď posloupnost dekadických číslic tvořící číslo, nebo posloupnost dekadických číslic, udávající základ číselné soustavy, a po ní znak \texttt{r} a posloupnost číslic z~oné číselné soustavy, písmena A až z~bez ohledu na~velikost se považují za~číslice 10 až 35. Lze používat i soustavy o základu vyšším než 36, ale není v~nich možné zapsat všechna čísla protože pro~ně nejsou další číslice.
\item \textbf{Literál reálného čísla} začíná jako literál celého čísla, ale pokračuje desetinnou tečkou po které můžou následovat další číslice. Rovněž lze používat libovolnou číselnou soustavu.
\item \textbf{Literál logické hodnoty} je \texttt{1b} pro~pravdu a \texttt{0b} pro~nepravdu. Je to totiž kratší než tradiční \texttt{true} a \texttt{false} a navíc je zápis obou hodnot stejně dlouhý.
\item \textbf{Literál textu} je posloupnost znaků v~uvozovkách. Patří do~něj všechny znaky mezi~nimi, tedy i konce řádků a speciální znaky, ale je-li potřeba zapsat znak uvozovek, musí se zdvojit, aby nedošlo k~ukončení literálu.
\item \textbf{Operátor} je libovolná poslounost znaků \uv{\texttt{+-*/\%\&|\textasciitilde\circumflex<>=!}}. Jestli dotyčný operátor opravdu existuje a lze jej aplikovat na~tyto argumenty se posoudí až při~vyhodnocování sémantiky.
\item \textbf{Oddělovač} je jeden znak; některý ze znaků \uv{\texttt{()[]\{\}.:}}.
\end{itemize}

\subsubsection{Příklad}
\begin{verbatim}
# Toto je komentář
# Deklarace konstant
a = 1103515245
# 12345 zapsané v~osmičkové soustavě
c = 8r30071
# Tady využíváme literály v~soustavě o základu 256
# Číslice 36 až 255 neexistují, ale toto číslo je nepotřebuje
m = 256r10000
# Pokud je poslední příkaz ve~funkci vyhodnocení výrazu,
# vrátí se hodnota toho výrazu
next = fun[x: int] int {
    x * a + c % m
}
# Zápis funkce s~více argumenty
mocnina = fun[zaklad exp: int] int {
    result = 1
    while exp > 0 {
    	result = result * zaklad
    	exp = exp - 1
    }
    return result
}
\end{verbatim}

\subsection{Kontext}
Každé jméno v~kódu překladač vyhodnoduje v~určitém kontextu, který zahrnuje všechna jména, jež na~daném místě v~kódu označují nějakou proměnnou, konstantu či jmenný prostor. Proměnné i konstanty jsou jména zastupující určitou hodnotu. Jmenný prostor je kolekce obsahující další proměnné, konstanty a jmenné prostory, k~nimž se přistupuje skrz onen jmenný prostor.

Proměnná se vytváří uvnitř bloku příkazů přiřazením hodnoty určitému jménu, jenž na~daném místě ještě nic neoznačuje, a je přístupné až do~konce toho bloku. Argumenty lambda funkcí jsou rovněž proměnné, jejichž hodnoty se nastaví při~zavolání funkce na~hodnoty předaných argumentů a jsou dostupné uvnitř celé lambda funkce. Proměnnou jde změnit tím, že se do~ní přiřadí nová hodnota; ta musí mít stejný typ jako předchozí hodnota proměnné, kterou nová hodnota nahradila.

Konstanta se deklaruje buď přiřazením na~úrovni souboru (vně všech bloků), tehdy je dostupná od přiřazení až po konec souboru, nebo vzniká v~cyklu \texttt{for} a je přístupná uvnitř jeho bloku. Přiřadíme-li na~úrovni souboru konstantě lambda funkci, bude konstanta přístupná v~celém souboru, nejen po přiřazení. Jednou vytvořené konstantě už není možné přiřadit novou hodnotu a není možné vytvořit ani jmenný prostor se stejným názvem, ale konstanta vytvořená v~cyklu \texttt{for} má novou hodnotu při~každé iteraci.

Jmenný prostor vzniká při~importu z~jiného souboru a obsahuje všechny exportované konstanty a jmenné prostory deklarované v~onom souboru. Od dotyčného importu do~konce souboru přes něj lze přistupovat k~jeho prvkům pomocí syntaxe \texttt{prostor.prvek}. Obsahuje-li jmenný prostor další exportované jmenné prostory, lze přístupy zřetězit a přistupovat i k~jejich prvkům: \texttt{prostor1.prostor2.prostor3.prvek}. Kruhové závislosti mezi~soubory jsou zakázané.

\subsection{Typový systém}
Zyba je staticky typovaný jazyk se silnou typovou kontrolou; typy všech hodnot tedy musí být známy v~čase překladu a nemůžou se implicitně konvertovat. Pro~konverzi hodnoty z~jednoho typu na~druhý lze použít zabudované metody, pokud programátor nechce konverzi implementovat ručně. Určitou zvláštností Zyby je to, že v~místě, kde je očekáván typ, se dá použít libovolná hodnota onoho typu (např. hodnota \texttt{3+4} se dá použít jako typ \texttt{int}). Jména \texttt{int}, \texttt{bool} a další totiž ve~skutečnosti znamenají pouze výchozí hodnoty těchto typů, nikoliv typy samotné (např. \texttt{int} je totéž jako \texttt{0}). Samozřejmě by se při~psaní kódu měly používat názvy typů na~místech, kde je očekáván typ, a hodnoty na~místech, kde je očekávána hodnota; je sice možné to nedodržovat, ale vzniklý kód bude matoucí. Tato zvláštnost ale má i některé výhody, protože jednak činí jazyk jednodušším a navíc umožňuje vytvářet typové aliasy úplně stejně jako se vytváří proměnné a konstanty. Ostatně tento přístup má určitou paralelu v~chování některých objektově orientovaných jazyků (např. C++\cite{cpp}), které umožňují volat statické metody jak na~třídě, tak i na~objektu; i v~nich se tedy dá někdy použít hodnota (objekt) na~místo typu (třídy).

\subsubsection{Datové typy}
\begin{tabular}{|m{3.5cm} m{6cm} m{5cm}|}\hline
Název & Popis & Výchozí hodnota \\\hline
\texttt{void} & Prázdná hodnota, návratový typ funkcí, které nic nevracejí & Bezvýznamná hodnota\\\hline
\texttt{int} & Celé číslo, 32bitové nebo 64bitové v~závislosti na~systému & Nula (\texttt{0})\\\hline
\texttt{real} & Reálné číslo & Nula jako reálné číslo (\texttt{0.0})\\\hline
\texttt{bool} & Logická hodnota & Nepravda (\texttt{0b})\\\hline
\texttt{text} & Textový řetězec & Prázdný text (\texttt{\quote{}})\\\hline
\texttt{db} & Připojení k~databázi & Nezahájené připojení\\\hline
\texttt{T.list} & Pole hodnot typu \texttt{T} & Prázdné pole\\\hline
\texttt{T.dict[K]} & Slovník s~klíči typu \texttt{K} (musí být \texttt{int} nebo \texttt{text}) a hodnotami typu \texttt{T} & Prázdný slovník\\\hline
\texttt{T.fun[A\textsubscript{0} \ldots{} A\textsubscript{m}]} & Funkce přebírající argumenty typů \texttt{A\textsubscript{0}} až \texttt{A\textsubscript{m}} a vracející \texttt{T} & Funkce přebírající libovolné argumenty typů \texttt{A\textsubscript{0}} až \texttt{A\textsubscript{m}} vracející vždy výchozí hodnotu typu \texttt{T}\\\hline
\texttt{\{N\textsubscript{0} T\textsubscript{0} \ldots{} N\textsubscript{m} T\textsubscript{m}\}} & Záznam s~prvky, jejichž názvy jsou \texttt{N\textsubscript{0}} až \texttt{N\textsubscript{m}} a typy jsou \texttt{T\textsubscript{0}} až \texttt{T\textsubscript{m}} & Záznam se všemi prvky nabývajícími svých výchozích hodnot\\\hline
\end{tabular}

Funkce se vytváří pomocí syntaxe lambda funkce, jež specifikuje typy argumentů a návratové hodnoty a obsahuje kód, který se při~zavolání funkce vykoná. Pokud se program dostane k~příkazu \texttt{return}, funkce skončí a vrátí hodnotu předanou tomuto příkazu. Poslední výraz ve~funkci, jejíž návratový typ není \texttt{void}, musí být \texttt{return} nebo vyhodnocení výrazu, které na~tomto místě znamená, že funkce výsledek tohoto výrazu vrátí. Funkce se volá pomocí hranatých závorek, předají se jí argumenty odpovídajících typů a výsledkem volání funkce je hodnota, již funkce vrátila. Každý soubor může obsahovat funkci s~názvem \texttt{main}, všechny tyto funkce se po spuštění zavolají, přičemž je zaručeno, že \texttt{main} z určitého souboru se zavolá až po zavolání funkcí z importovaných souborů a jejich závislostí.

Záznam se vytváří stejným způsobem jako se značí typ záznamu, jen se píší hodnoty jeho prvků a ne typy. Zápis \texttt{\{N\textsubscript{0} H\textsubscript{0} \ldots{} N\textsubscript{m} H\textsubscript{m}\}} vytvoří záznam, jehož prvky budou mít názvy \texttt{N\textsubscript{0}} až \texttt{N\textsubscript{m}} a nabývají hodnot \texttt{T\textsubscript{0}} až \texttt{T\textsubscript{m}}. K~prvkům záznamu se přistupuje pomocí tečky (např. \texttt{zaznam.prvek}), přičemž obsahuje-li záznam další záznamy, lze přístupy řetězit stejně jako u jmenných prostorů.

Pole a slovníky se vytváří zcela totožným způsobem, jakým se značí typ pole či slovníku, právě proto, že typ je skutečnosti sám jen prázdné pole či slovník. Při~vytváření lze rovnou přidat prvky tím, že se zabudované metodě \texttt{list} či \texttt{dict} předají další argumenty, což budou hodnoty oněch prvků. U slovníku se jako argumenty střídají klíče a prvky: \texttt{T.dict[K K\textsubscript{0} V\textsubscript{0} \ldots{} K\textsubscript{n} V\textsubscript{n}]}. Připojení k databázi se vytváří zabudovanou metodou \texttt{connect}, jež přebírá tři argumenty typu \texttt{text}: \textit{connection string} používaný PHP knihovnou PDO, uživatelské jméno a heslo.

Pole, slovníky i připojení k~databázi jsou tzv. referenční typy, což znamená, že do~proměnných se neukládá jejich hodnota, ale pouze reference na~místo v~paměti, kde je jejich hodnota uložená. Přiřadíme-li je proto do~více proměnných, budou všechny obsahovat reference na~stejné místo, takže změníme-li pomocí zabudovaných metod hodnotu na~jednom místě, projeví se tato změna ve~všech proměnných, které obsahují referenci na~toto místo.

\subsubsection{Zabudované metody a operátory}
Datové typy mají svoje zabudované metody a operátory. Obojí jsou zvláštní výrazy, jež něco vykonají a vrátí nějakou hodnotu, ale liší se syntaxí. Syntaxe volání zabudovaných metod i operátorů již byla popsáná výše. Operátory logické konjunkce a disjunkce svoje argumenty vyhodnocují líně, tedy plyne-li ze znalosti prvního argumentu znalost výsledku, druhý argument se nevyhodnoduje. Operátory se dělí na~aritmetické, bitové, logické, porovnávací a složené. Aritemetické operátory přebírají dva argumenty typu \texttt{int} nebo \texttt{real}, přičemž oba argumenty nemusí mít stejný typ. Patří mezi~ně:

\begin{tabular}{|m{1.5cm} m{5cm} m{8cm}|}\hline
Operátor & Výsledek & Typ\\\hline
\texttt{+} & Součet & \multirow{5}{*}{\texttt{int}, jsou-li argumenty \texttt{int}, jinak \texttt{real}}\\
\texttt{-} & Rozdíl & \\
\texttt{*} & Součin & \\
\texttt{**} & Mocnina & \\
\texttt{\%} & Zbytek po dělení (pro záporné dělence je záporný) & \\\hline
\texttt{/} & Podíl & \texttt{real}\\\hline
\texttt{//} & Podíl (zaokrouhlený k~nule) & \texttt{int}\\\hline
\end{tabular}

Bitové operátory přijímají dva argumenty typu \texttt{int} a vrací rovněž \texttt{int}.

\begin{tabular}{|m{1.5cm} m{5cm} m{8cm}|}\hline
Operátor & Výsledek & Typ\\\hline
\texttt{\&} & Bitová konjunkce & \texttt{int}\\\hline
\texttt{|} & Bitová disjunkce & \texttt{int}\\\hline
\texttt{\^} & Bitová nonekvivalence & \texttt{int}\\\hline
\end{tabular}

Logické operátory přebírají dva argumenty typu \texttt{bool} a vrací \texttt{bool}, přičemž logická konjunkce a disjunkce svoje argumenty vyhodnocují líně, což znamená, že plyne-li ze znalosti prvního argumentu znalost výsledku, druhý argument se nevyhodnoduje.

\begin{tabular}{|m{1.5cm} m{5cm} m{8cm}|}\hline
Operátor & Výsledek & Typ\\\hline
\texttt{\&} & Logická konjunkce & \texttt{bool}\\\hline
\texttt{|} & Logická disjunkce & \texttt{bool}\\\hline
\texttt{\^} & Logická nonekvivalence & \texttt{bool}\\\hline
\end{tabular}

Porovnávací operátory zjišťují buď rovnost nebo uspořádání dvou hodnot. Ty pro~zjišťování rovnosti přebírají dva argumenty stejného typu, který ovšem není funkce, \texttt{void}, \texttt{real}, \texttt{db}, ani pole, slovník či je obsahující. Zjišťování rovnosti u reálných čísel není povolené, protože kvůli jejich reprezentaci v~počítači tato operace může dávat špatné výsledky. Pole, slovníky a záznamy se považují za~rovné, pokud jsou si rovné všechny jejich prvky. Operátory zjišťující uspořádání přebírají buď dvě hodnoty typu \texttt{text}, nebo dvě čísla, z~nichž každé je typu \texttt{int} či \texttt{real}.

\begin{tabular}{|m{1.5cm} m{5cm} m{8cm}|}\hline
Operátor & Výsledek & Typ\\\hline
\texttt{==} & Rovnost & \texttt{bool}\\\hline
\texttt{!=} & Nerovnost & \texttt{bool}\\\hline
\texttt{<} & Menší než & \texttt{bool}\\\hline
\texttt{>} & Větší než & \texttt{bool}\\\hline
\texttt{<=} & Menší nebo rovno & \texttt{bool}\\\hline
\texttt{>=} & Větší nebo rovno &\texttt{bool}\\\hline
\end{tabular}

Složené operátory pracují s~poli, slovníky, záznamy a typem \texttt{text}. Operátory pracující s~poli a slovníky přebírají dva argumenty stejného typu, nazvěme jej \texttt{T.list}, jde-li o pole, a \texttt{T.dict[K]}, jde-li o slovník. Dojde-li při~průnicích nebo sjednoceních slovníků a záznamů ke~kolizi dvou prvků se stejným klíčem nebo názvem, započítá se do~výsledku prvek z~pravého argumentu.

\begin{tabular}{|m{1.5cm} m{5cm} m{8cm}|}\hline
Operátor & Výsledek & Typ\\\hline
\texttt{+} & Spojení textů & \texttt{text}\\\hline
\texttt{+} & Spojení polí & \texttt{T.list}\\\hline
\texttt{|} & Sjednocení slovníků & \texttt{T.dict[K]}\\\hline
\texttt{\&} & Průnik slovníků & \texttt{T.dict[K]}\\\hline
\texttt{|} & Sjednocení záznamů & Typ záznamu, který obsahuje všechny prvky, které byly alespoň v~jednom ze záznamů\\\hline
\texttt{\&} & Průnik záznamů & Typ záznamu, který obsahuje všechny prvky, které byly v~obou záznamech\\\hline
\end{tabular}

\section{Konstrukce překladače}
Překladač je napsaný v~jazyce Haskell a jeho kód je rozdělený do~několika modulů.
\begin{figure}[H]
\begin{center}
	\begin{tikzpicture}[node distance=2cm]
		\node (functions) [block] {Functions};
		\node (fallible) [block, right of=functions, xshift=2cm] {Fallible};
		\node (scope) [block, below of=functions, xshift=2cm] {Scope};
		\node (parser) [block, above of=functions] {Parser};
		\node (lexer) [block, above of=fallible] {Lexer};
		\node (codegen) [block, left of=functions, xshift=\minus4cm] {Codegen};
		\node (language) [block, above of=codegen, xshift=2cm]{Language};
		\node (semantics) [block, below of=functions, xshift=\minus4cm] {Semantics};
		\node (loader) [block, below of=semantics, xshift=2cm] {Loader};
		\node (main) [block, left of=loader, xshift=\minus2cm] {Main};
		\draw [arrow] (fallible) -- (lexer);
		\draw [arrow] (fallible) -- (parser);
		\draw [arrow] (functions) -- (parser);
		\draw [arrow] (lexer) -- (parser);
		\draw [arrow] (functions) -- (language);
		\draw [arrow] (fallible) -- (language);
		\draw [arrow] (fallible) -- (scope);
		\draw [arrow] (language) -- (scope);
		\draw [arrow] (parser) -- (scope);
		\draw [arrow] (parser) -- (semantics);
		\draw [arrow] (scope) -- (semantics);
		\draw [arrow] (language) -- (semantics);
		\draw [arrow] (functions) -- (semantics);
		\draw [arrow] (fallible) -- (semantics);
		\draw [arrow] (parser) -- (loader);
		\draw [arrow] (scope) -- (loader);
		\draw [arrow] (language) -- (loader);
		\draw [arrow] (functions) -- (loader);
		\draw [arrow] (fallible) -- (loader);
		\draw [arrow] (semantics) -- (loader);
		\draw [arrow] (language) -- (codegen);
		\draw [arrow] (semantics) -- (codegen);
		\draw [arrow] (parser) -- (codegen);
		\draw [arrow] (functions) -- (codegen);
		\draw [arrow] (loader) -- (main);
		\draw [arrow] (codegen) -- (main);
		\draw [arrow] (fallible) -- (main);
	\end{tikzpicture}
	\caption{Graf závislostí mezi~moduly}
	\label{projects}
\end{center}
\end{figure}

\subsection{Přehled modulů}
Modul \texttt{Functions} zahrnuje pouze pomocné funkce, které nemají nic společného s~logikou tohoto konkrétního programu a pracují s~textem, seznamy a monádami. Slouží ke~zkrácení kódu v~ostatních modulech tím, že abstrahují chování, jež se v~nich opakuje.

Modul \texttt{Fallible} obsahuje stejnojmennou monádu, její transformátor a funkce pro~práci s~nimi. Monáda \texttt{Fallible} je alias pro~\texttt{Either String} a obsahuje buď hodnotu nějakého výsledku, nebo text s~chybovou hláškou. Její spojování je definováno tak, že spojení chybové hodnoty s~další funkcí bude tuto funkci ignorovat a vrátí původní chybovou hodnotu. Pokud v~některé funkci při~překladu programu dojde k~chybě, což může nastat jen tehdy, když je program špatně napsaný, vrátí příslušná funkce hodnotu \texttt{Fallible} reprezentující chybu a vzhledem k~vlastnostem spojení se další části překladu neprovedou a výsledkem programu bude chyba. Dále je zde definována monáda \texttt{FallibleIO}, což je alias pro~výsledek monádového transformátoru, který zabalí monádu \texttt{IO} do~\texttt{Fallible}.

Modul \texttt{Lexer} implementuje lexikální analyzátor pro~Zybu. Jeho funkce \texttt{tokenize} přebírá \texttt{String} a vrací \texttt{Fallible [(Integer, Token)]}, z~čehož každá dvojice reprezentuje token a číslo řádku, na~němž se token nacházel. Čísla řádků se používají při~generování chybových hlášek. \texttt{Token} je algebraický datový typ, jehož konstruktory odpovídají jednotlivým typům tokenů zmiňovaných v~sekci Syntaxe.

Modul \texttt{Parser} provádí syntaktickou analýzu programu. Obsahuje funkci \texttt{parse}, jež dostane na~vstupu \texttt{String}, rozdělí jej na~tokeny pomocí funkce \texttt{tokenize}, sestaví syntaktický strom a vrátí \texttt{Fallible File}. Typ \texttt{File}obsahuje seznam trojic \texttt{(Integer, Visibility, Declaration)}, kde první prvek je opět číslo řádku, druhý, \texttt{Visibility}, udává, zda je deklarace exportovaná, a třetí je samotná deklarace. Dále modul obsahuje algebraické datové typy modelující syntaktický strom \texttt{Declaration} pro~deklarace, \texttt{Statement} pro~příkazy, \texttt{Value} pro~výrazy a \texttt{Literal} pro literály. Mají konstruktory pro~každý typ syntaktické konstrukce zmíněné v~sekci syntaxe.

Modul \texttt{Language} definuje typy (algebraický typ \texttt{Type}) a zabudované metody a operátory (algebraický typ \texttt{Builtin}) jazyka Zyba. Kromě toho jsou v~něm obsaženy funkce pro~jejich vyhledávání a typovou kontrolu při~operacích s~nimi.

Modul \texttt{Scope} definuje stejnojmenný typ, který reprezentuje kontext pro~proměnné, konstanty a jmenné prostory. Modul ovšem neexportuje konstruktor typu \texttt{Scope}, takže z~jiných modulů lze kontext číst a měnit jen pomocí dalších funkcí z~tohoto modulu.

Modul \texttt{Semantics} se zabývá sémantickou analýzou. Jsou tu definovány typy \texttt{Value} a \texttt{Statement}, jejichž hodnoty se vytváří ze stejnojmenných typů z~modulu \texttt{Parser} po~sémantické analýze. Hlavní funkce \texttt{analyse} přijímá hodnotu typu \texttt{File} se syntaktickým stromem analyzovaného souboru, mapu \texttt{Map String Scope}, obsahující kontexty s~deklaracemi exportovanými z~již analyzovaných souborů překládaného programu, a \texttt{String} obsahující cestu k~právě analyzovanému souboru; návratová hodnota je typu \texttt{Fallible (Scope, [Declaration])}. První prvek dvojice je kontext s~exportovanými deklaracemi, druhým prvkem je seznam deklarací; v seznamu však jsou jen hodnoty a nikoliv importy, protože seznam se používá jen při~generování kódu pro~deklarované hodnoty. \texttt{Declatation} se definuje jako \texttt{(String, (Value, Type))}, což jsou název, hodnota a typ deklarované hodnoty.

Modul \texttt{Loader} zajišťuje správu importů a spojení syntaktické a sémantické analýzy. Obsahuje funkci \texttt{load}, jež dostane na~vstupu cestu k~hlavnímu souboru a vrátí jako výsledek \texttt{FallibleIO ([String], [(String, [Declaration])])}. První prvek vnější dvojice je seznam obsahů všech importovaných PHP souborů. Druhý prvek je seznam dvojic, z~nichž každá obsahuje název a seznam deklarací hodnot jednoho souboru Zyby. Zyba soubory jsou seřazeny v~takovém pořadí, že pokud modul A závisí na~modulu B, bude v~seznamu modul B před modulem A. Z~toho taky plyne, že není možné mít kruhové závislosti a v~případě výskytu takové závislosti vrátí funkce \texttt{load} hodnotu chyby.

Modul \texttt{Codegen} završuje kompilaci, neboť jeho funkce \texttt{gen} vygeneruje PHP kód na~základě výstupu z~funkce \texttt{load}. V této funkci už nemůže dojít k chybě, protože program prošel všemi fázemi kontroly, takže návratová hodnota funkce je pouze \texttt{String}.

Modul \texttt{Main} poté ve~funkci \texttt{main} přečte dva argumenty z~příkazové řádky -- cestu ke~vstupnímu Zyba souboru a výstupnímu PHP souboru. Zavolá funkci \texttt{load} s~cestou ke~vstupnímu souboru, její výsledek předá funkci \texttt{gen} a vzniklý PHP kód zapíše do~zadaného souboru. Dojde-li k chybě, nic se do zadaného souboru nezapíše a chybová hláška bude vypsána na standardní chybový výstup.

\subsection{Ukázka překladu}

\section{Závěr}
Navrhl jsem programovací jazyk Zyba a zkonstruován jeho překladač. Jazyk lze použít pro~psaní webových stránek, přičemž funkce pro~interakci s~okolím jsou buď obsaženy jako zabudované metody (práce s~databází) nebo je lze importovat z~PHP. Kód v~Zybě napsaný je dobře čitelný, překladač provádí statickou typovou kontrolu a systém importů a exportů je poměrně silný, třebaže by bylo možné implementovat ještě kruhové importy. Hlavní slabinou jazyka je chybějící podpora pro~generické programování, což znemožňuje tvorbu funkcí, které by pracovaly s~libovolnými daty (řazení, přístup k~databázi) a je tak nutné pro~ně buď vytvořit v~jazyce zabudované metody, nebo mít jednoúčelové funkce. To jsou tedy možné směry pro~budoucí vývoj jazyka.

\newpage
\printbibliography[heading=bibintoc, title={Použitá literatura}]
\newpage
\end{document}